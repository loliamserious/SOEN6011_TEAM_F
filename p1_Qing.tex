\documentclass[12pt]{extarticle}
\usepackage[utf8]{inputenc}
\usepackage{cite}

\title{Tangent Function : f(x) = tan (x)}
\author{Qing Li}
\date{Student id: 40082701}

\begin{document}

\maketitle



Domain: all real numbers except pi/2 + k pi, k is an integer.\\

Co-domain: all real numbers.\\

Characteristic:\\

-Period = pi\\ 

-x intercepts: x = k pi , where k is an integer.\\ 

-y intercepts: y = 0 \\

-symmetry: since tan(-x) = - tan(x) then tan (x) is an odd function and its graph is symmetric with respect the origin. \\

-Intervals of increase/decrease: over one period and from -pi/2 to pi/2, tan (x) is increasing. \\

-Vertical asymptotes: x = pi/2 + k pi, where k is an integer. \\

Importance of tangent function:\\

Writing the numerators as square roots of consecutive natural numbers

{\displaystyle {\frac {\sqrt {0}}{2}},{\frac {\sqrt {1}}{2}},{\frac {\sqrt {2}}{2}},{\frac {\sqrt {3}}{2}},{\frac {\sqrt {4}}{2}}} 

provides an easy way to remember the value.\\


\end{document}
